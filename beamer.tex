\documentclass{beamer}

\usepackage[T1]{fontenc}
\usepackage[utf8x]{inputenc}
\usepackage[francais]{babel}
\usepackage[english]{babel}
\uselanguage{French}
\usepackage{dsfont}
\usepackage{bbold}
\usepackage{stmaryrd}
\languagepath{French}
\usepackage{xcolor}

\usepackage{pgf}
\usepackage{tikz}


\mode<presentation> {
\usetheme{AnnArbor}

\setbeamercovered{transparent}
\definecolor{rouge}{RGB}{255,0,0}
\definecolor{almond}{rgb}{0.94, 0.87, 0.8}
\definecolor{blue(pigment)}{rgb}{0.2, 0.2, 0.6}
\definecolor{or}{RGB}{255,204,0}
\definecolor{cerise}{rgb}{0.87, 0.19, 0.39}
\definecolor{fandango}{rgb}{0.71, 0.2, 0.54}
\setbeamercolor{structure}{fg=rouge,bg=or}
\setbeamercolor{titlelike}{fg=noir,bg=or}
\setbeamercolor{frametitle}{fg=blue,bg=or}
\setbeamercolor{item}{fg=rouge,bg=or}

}

%%%%%%%%%%%%%%%%%%%%%%%%%%%%%%%%%%%

%%%%%%%%%%%%%%%%%%%%%%%%%%%%%%%%%%%

\title{ Anova2F}
\author{\color{rouge} Khalifi Oumayma\\Walid Kandouci\\ Abdestar SAHBANE\\\color{fandango} M2 MIND ET BIOSTAT}

\institute{Faculté des sciences Montpellier}

\date{\today}




\begin{document}

% page titre
\begin{frame}
\titlepage
\end{frame}

% plan de la présentation
\begin{frame}
\frametitle{Summary}
\tableofcontents
\end{frame}
\section{Introduction}
\begin{frame}
\frametitle{Introduction}
We discussed in the previous paragraph the one-way ANOVA and its uses.
%that examines the influence of one categorical independent variables on another continuous dependent variable.\\
\\
In this paragraph, we will be looking at two-way ANOVA, an extension of the one-way ANOVA that examines the influence of two different categorical independent variables on one continuous dependent variable. \\
The two-way ANOVA not only aims at assessing the main effect of each independent variable but also if there is any interaction between them.
\end{frame}
\end{document}